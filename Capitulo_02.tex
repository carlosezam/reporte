\chapter{Descripción del proyecto}
En este capítulo se presenta la descripción del proyecto “Implementación de un algoritmo de sincronización de semáforos usando inteligencia artificial”, en donde se da a conocer la razón por la cual se decidió desarrollar la investigación, así como también, los objetivos que se deben cumplir, los problemas que resuelve el mismo, los alcances y limitaciones que se presentan.

\section{Problemas a resolver}
En la ciudad de Tapachula, no existen semáforos inteligentes que sean capaces de resolver los problemas que todos los días se presentan a consecuencia de la gran cantidad de automóviles que circulan en las calles.

Debido a ello en un futuro se pretende la creación de estos semáforos, pero para su desarrollo es necesario el uso de un algoritmo de sincronización que permita la comunicación entre semáforos de avenidas intersectadas y decida el estatus de cada uno de ellos.

\section{Alcances y limitaciones}
\subsection*{Alcances del proyecto}
\begin{itemize}
\item El algoritmo contará con un tiempo de respuesta en el orden de milisegundos para su implementación en el control de semáforos.
\end{itemize}

\subsection*{Limitaciones del proyecto}
\begin{itemize}
\item El algoritmo solo comunicará los semáforos de cuatro avenidas que se intersectan.
\end{itemize}

\section{Objetivos}

\subsection*{Objetivo general}
Implementar un algoritmo de sincronización de semáforos utilizando inteligencia artificial.

\subsubsection*{Objetivos específicos}
\begin{itemize}
\item Seleccionar la técnica de inteligencia artificial que resuelva la sincronización entre los semáforos.
\item Implementar un algoritmo que use la técnica seleccionada y que determine el estatus de cada semáforo y el tiempo que permanecerá dicho estatus.
\end{itemize}

\section{Justificación}
Las ciudades son lugares en donde se realiza una alta actividad económica y en donde transportarse forma parte de la vida cotidiana de las personas; mientras que para las empresas representa una parte esencial de su operación diaria. Estos factores hacen que en las calles principales de la ciudad exista tráfico vehicular y aumente la contaminación del medio ambiente, lo cual es perjudicial para las personas y el ambiente mismo.

Tomando en cuenta los puntos antes mencionados, se considera importante, que en la ciudad de Tapachula, Chiapas, se debe de contar con un sistema inteligente que permita controlar de manera eficiente el flujo vehicular y peatonal, para lograr este objetivo será necesario que el sistema sea capaz de realizar el conteo de los automóviles de una forma rápida, logrando con ello tener un control que permita administrar el tiempo de espera para ceder el paso a los automóviles y así mismo a los peatones.
