\chapter[Algoritmo de detección de vehículos]{Implementación en C++ de un algoritmo de detección de vehículos}\label{apendice:b}
Del resultado de la investigación realizada en \cite{ittap} se rescata el algoritmo de visión computacional seleccionado para la detección de automóviles. Dicho trabajo arroja un algoritmo implementado en \emph{Python} que, según los resultados de la propia investigación, ha probado tener una eficacia (con buena iluminación) de hasta un 95\%.

El tiempo de ejecución del algoritmo sugerido por \emph{Estrada \& Recinos \& Vidal (2017)}: ``Detección de vehículos mediante imagen de fondo '' es de 1.5 segundos (implementado en Python), sin embargo al implementarse en C++ alcanza un tiempo de 0.03 segundos, es decir, 3 centésimas de segundo. 

\textbf{Requerimientos}\\
El algoritmo en su versión C++ requiere de un compilador que soporte el estándar ISO C++11 además de tener compilado e instalado la librería para visión por computadora OpenCV en su versión 2.x. \\\\
\textbf{Código Fuente}\\
\lstinputlisting[
style=ezam,
caption={[Implementación en C++] Implementación en C++ del algoritmo de detección de vehículos}
]{algoritmo.cpp}
