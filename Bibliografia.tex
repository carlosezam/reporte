\begin{thebibliography}{99}
	
	\bibitem{garciag} García, G. (2017). Un enfoque de semáforo inteligente utilizando algoritmos de visión computacional en una intersección aislada para optimizar el flujo vehicular (Tesis de maestría). Centro de Investigación en Inteligencia Artificial (CIIA) de la Universidad Veracruzana, Xalapa de Enríquez, Veracruz, México. 
	
	\bibitem{alvaroer} Alvaro E., R. De Somocurcio S. (2008). Control de tráfico vehicular automatizado utilizando Lógica Difusa. Universidad Ricardo Palma, Lima, Perú.
	
	\bibitem{bencesramos} Bances, M., Ramos, M. (2014). Semáforos Inteligentes para la regulación del tráfico vehicular. Rev. Ingeniería: Ciencia, Tecnología e Innovación. Vol. 1 (No. 1). pp. 37-45. .
	
	\bibitem{MoraleslGonzales} Morales, L. Rafael., Gonzáles, S. Juan. (2013) \emph{Control de tráfico vehicular por medio de semáforos inteligentes}. Universidad de Rafael Urdaneta, República Bolivariana de Venezuela.  
	
	\bibitem{Hernandezca} Hernández, C.A., Salcedo, O., \& Pedraza, L.F. (2007). Modelo de Semaforización Inteligente para la Ciudad de Bogotá. Revista Científica y Tecnológica de la Facultad de Ingeniería, Universidad Distrital Francisco José de Caldas, Vol. 11(No.2). 61-69. 
	
	\bibitem{ittap} Estrada, E.K., Recinos, H., \& Vidal, G. Y. (2017). Selección de un algoritmo de visión por computadora para la detección de vehículos. Instituto Tecnológico de Tapachula, Tapachula, Chiapas, México.
	
	\bibitem{duarte} Oscar G. Duarte (1999). Sistemas de lógica difusa. Fundamentos. \textit{Revista Ingeniería e Investigación, No. 42, pp. 22.}
	
	\bibitem{sedesolpa} Secretaría de Desarrollo Social (SEDESOL). (1994). Programa de Asistencia técnica en transporte urbano para las ciudades medias mexicanas. Manual Normativo, Tomo XII. Estudios de Ingeniería de Tránsito. México.
	
	\bibitem{minusterioti} Ministerio de Transporte e Infraestructura. (2008). Manual para la Revisión de Estudios de Tránsito. Realización de Manuales Técnicos para la Revisión y Aprobación de Estudios y Diseños de Carreteras. Managua, Nicaragua.
	
	\bibitem{arandia} Juan Gabriel Tapia Arandia, Romel Daniel Veizaga Balta (2006).
		\emph{Apoyo didáctico para la enseñanza y aprendizaje de la asignatura de ingeniería de tráfico}.
		(Trabajo para optar al diploma de Licenciatura en Ingeniería Civil).
		Universidad Mayor De San Simón, Cochabamba, Bolivia.
	
	\bibitem{zadehfs} Zadeh, A.L. (1965). Fuzzy sets \emph{Information and Control}, vol. 8, pp. 338-353.
	
	\bibitem{zadehnewapproach} Zadeh, A.L. (1973). Outline of a New Approach to the Analysis of Complex Systems and Decision Processes. \emph{IEEE Transactions on Systems, Man, and Cybernetics}, Vol. SMC-3, (No. 1). 28-44.
	
	\bibitem{zadehlinguisticv} Zadeh, A.L. (1975). The Concept of a Linguistic Variable and its Application to Approximate Reasoning-I. \emph{Information Sciences}. Vol. 8. 199-249.
	
	\bibitem{ponce} Ponce Cruz P. (Primera Edición). (2010).
		\emph{Inteligencia Artificial con Aplicaciones a la Ingeniería}. México: Alfaomega.
	
	\bibitem{amador} Amador Hidalgo L. (Primera Edición). (1997).
		\emph{Inteligencia Artificial y Sistemas Expertos}. Córdoba: Universidad de Córdoba.
		
	\bibitem{alfonsovr2013} Alfonso V. Rivera. (2013). \emph{Controladores difusos aplicados a convertidores DC/DC} (Tesis de maestría). Universidad Autónoma de Aguascalientes, Aguascalientes, Ags.
	
	\bibitem{carlosgm} Carlos G. Morcillo. \emph{Lógica Difusa, una introducción práctica}. E-mail: Carlos.Gonzales@uclm.es
	
	\bibitem{josecarlos} José Carlos. \emph{Control Neuro-Difuso Aplicado a una Grúa Torre}. Recuperado de Tesis Digitales UNMSM
	
	\bibitem{kevinstephe} Kevin M. Passino, Stephen Yurkovick, (1997). \emph{Fuzzy Control}. California, Berkeley: ADDISON-WESLEY
	
	\bibitem{MarinIA} Marín, M. R., Palma, M. J. (2008). Inteligencia Artificial: Métodos, técnicas y aplicaciones. España: Mcgraw-Hill/Interamericana De España, S. A. U.
	
	\bibitem{IssaiGalvan} Isasi, V. P., Galván, L. I. (2004). Redes de Neuronas Artificiales: Un enfoque práctico. Madrid, España: Pearson Educación, S.A.
	
	\bibitem{caf} Observatorio de movilidad urbana. (2013). \emph{Qué es movilidad urbana}. Recuperado Diciembre 20, 2018 en: https://www.caf.com/es/actualidad/noticias/2013/08/que-es-movilidad-urbana/?parent=14062.
	
\end{thebibliography}

